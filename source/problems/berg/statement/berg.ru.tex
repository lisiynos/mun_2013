\gdef\thisproblemauthor{Иван Казменко}
\gdef\thisproblemdeveloper{Иван Казменко}
\begin{problem}{Семья Берг и секретный язык}
{berg.in}{berg.out}
{2 секунды}{256 мебибайт}{}

Семья Берг "--- папа, мама и четверо детей "--- живёт в кирпичном доме
в самом центре городка Штиль.
Дети часто оставляют друг другу записки в разных местах:
под входной дверью, на подоконнике, в дупле растущего во дворе дуба.

Недавно дети обнаружили, что соседские ребята "--- семья Борг "--- тайком
читают их переписку!
Кроме того, в старых проверенных местах появляются фальшивые записки.
Например, в записке Анна Берг просит свою сестру Беллу прийти к двум
часам дня на главную площадь города "--- но, когда Белла приходит
в назначенный срок, Анны там не оказывается, а позже выясняется,
что Анна вообще не оставляла такого сообщения...

Чтобы обезопасить себя от проделок семьи Борг, ребята решили придумать
секретный язык, на котором можно оставлять друг другу сообщения так,
чтобы никто посторонний не мог ни прочитать их, ни подделать.
Прежде всего дети договорились в своих сообщениях не ставить пробелы
и знаки препинания, а ограничиться только маленькими буквами
английского алфавита.
Кроме того, чтобы зашифровать такую последовательность букв, между каждыми
двумя соседними буквами решили вставлять английскую букву <<\texttt{a}>>.
Это значит, что сообщение <<\texttt{iamberg}>>, например, следует записывать
на секретном языке как <<\texttt{iaaamabaearag}>>.

Читать записки стало непросто!
По тексту очередной записки восстановите незашифрованное сообщение
или определите, что записка подозрительная, потому что не может являться
результатом шифрования, о котором договорились дети.

% new page here if \newpageafterlegend is defined, requires e-tex
\ifdefined\newpageafterlegend\newpage\fi

\InputFile

В единственной строке ввода задан текст найденной записки.
Он целиком состоит из маленьких букв английского алфавита.
Длина этого текста "--- от $1$ до $100$ букв включительно.

\OutputFile

Если заданный текст записки может быть результатом шифрования,
выведите в единственной строке исходное незашифрованное сообщение.
В противном случае выведите слово <<\texttt{Suspicious!}>>, означающее,
что записка подозрительная.

\Examples

\begin{example}
\exmp{
iaaamabaearag
}{%
iamberg
}%
\exmp{
abacaba
}{%
Suspicious!
}%
\end{example}

\Explanations

В первом примере ответ <<\texttt{iamberg}>> правильный,
потому что после вставки в него букв <<\texttt{a}>> по правилам шифрования
получается в точности заданный текст.

Во втором примере записка подозрительная, поскольку, например,
вторая буква в зашифрованном сообщении, если она есть, должна быть буквой
<<\texttt{a}>>, вставленной между первой и второй буквами исходного сообщения.
Обратите внимание: первая буква <<\texttt{S}>> должна быть
заглавной, а в конце должен стоять восклицательный знак.

\end{problem}
