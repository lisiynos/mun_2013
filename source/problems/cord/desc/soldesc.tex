Для каждого забора в отдельности проверим, что точка находится внутри него.

Для окружности это просто проверка того, что расстояние между точкой
и центром окружности меньше её радиуса.

Принадлежность точки прямоугольнику со сторонами, параллельными
осям координат, проверяется отдельно по каждой координате.

В случае треугольника (назовём его $\Delta ABC$) это означает,
что точка (назовём её $O$) лежит справа от трёх векторов $\overline{AB}$,
$\overline{BC}$ и $\overline{CA}$, или же слева ото всех трёх этих векторов.
Это можно проверить, вычислив три косых произведения
$\overline{AB} \wedge \overline{AO}$,
$\overline{BC} \wedge \overline{BO}$ и
$\overline{CA} \wedge \overline{CO}$ и посмотрев на их знак.
Косое произведение векторов $(x_1, y_1)$ и $(x_2, y_2)$ равно
$x_1 \cdot y_2 - x_2 \cdot y_1$.
