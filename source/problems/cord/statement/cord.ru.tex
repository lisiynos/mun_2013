\gdef\thisproblemauthor{Иван Казменко}
\gdef\thisproblemdeveloper{Иван Казменко}
\begin{problem}{Рассеянный Корд и три забора}
{cord.in}{cord.out}
{2 секунды}{256 мебибайт}{}

Семья Корд "--- Карл и его сестра Клара "--- живёт в бревенчатом доме
на окраине городка Штиль.
По утрам Карл отправляется через поле в старую Лабораторию,
а Клара "--- в Мастерскую на главной улице городка.
Вечером оба возвращаются домой.

Карл Корд "--- очень рассеянный.
Нередко он забывает, как добираться домой из Лаборатории.
Проплутав до ночи в поле, он расстилает в траве свой спальный мешок
и спит до утра, после чего опять идёт в Лабораторию.

Клара боится, как бы на Карла ночью не напали дикие звери.
Поэтому она построила в поле, где обычно теряется Карл, три складных забора.
Один забор имеет форму окружности, второй "--- прямоугольника,
а третий "--- треугольника.
В обычное время заборы скрыты в земле.
Когда брат не приходит вечером домой, Клара достаёт Пульт и включает заборы:
будучи включёнными, они медленно поднимаются из земли и могут защитить Карла,
если он оказался внутри какого-то из них.
Утром Клара выключает заборы, они прячутся обратно в землю,
и Карл идёт по своим делам.

Однажды утром Пульт сломался, и Клара не смогла выключить заборы.
Проснувшись, Карл решил не ждать, пока заборы скроются в земле,
а сразу же идти в Лабораторию.
При этом некоторые заборы, возможно, придётся перелезать.
Карл связался с Картографом, и тот построил для Карла Карту,
на которой схематично показаны поле, сам Карл, три забора и Лаборатория.

Карта представляет собой плоскость, на которой введена прямоугольная
декартова система координат.
В начале координат, то есть в точке $(0, 0)$, находится Карл,
а в точке $(1000, 1000)$ расположена Лаборатория, в которую ему надо попасть.
Забор в форме окружности задан координатами центра и радиусом этой окружности.
Стороны прямоугольного забора параллельны координатным осям, поэтому
он задан координатами двух противоположных углов.
Треугольный забор задан координатами трёх своих углов.

Помогите Карлу по данным Карты выяснить, какое минимальное количество раз
ему придётся перелезать через забор, чтобы добраться до Лаборатории.

% new page here if \newpageafterlegend is defined, requires e-tex
\ifdefined\newpageafterlegend\newpage\fi

\InputFile

В первой строке ввода задано три целых числа $x$, $y$ и $r$ "--- координаты
центра забора в форме окружности и радиус этой окружности ($r > 0$).
Во второй строке задано четыре целых числа $x_1$, $y_1$, $x_2$ и $y_2$ "---
координаты левого нижнего и правого верхнего углов прямоугольного забора
($x_1 < x_2$, $y_1 < y_2$).
В третьей строке задано шесть целых чисел $x_A$, $y_A$, $x_B$, $y_B$,
$x_C$ и $y_C$ "--- координаты углов треугольного забора.
Гарантируется, что площадь фигуры, ограниченной каждым из заборов,
строго положительная, никакие два забора не пересекаются и не касаются.
Кроме того, ни один забор не проходит через начало координат "--- точку,
в которой изначально находится Карл.
Все заданные координаты и радиус не превосходят $100$ по абсолютной величине.

\OutputFile

В единственной строке выведите одно число "--- минимальное количество раз,
которые Карлу придётся перелезать через забор, чтобы добраться до Лаборатории.

\Examples

\begin{example}
\exmp{
5 5 5
4 4 6 6
1 1 1 0 0 1
}{%
0
}%
\exmp{
0 0 1
-2 -2 2 3
-9 -3 3 -3 3 8
}{%
3
}%
\end{example}

\Explanations

В первом примере Карл может просто обойти все заборы.

Во втором примере Карл находится внутри забора в форме окружности,
этот забор "--- внутри прямоугольного, а он, в свою очередь "--- внутри
треугольного.
Все три забора придётся перелезать, чтобы дойти до Лаборатории.

Обратите внимание: углы треугольного забора могут быть перечислены
как по часовой стрелке (как во втором примере),
так и против часовой стрелки (как в первом примере).

\end{problem}
