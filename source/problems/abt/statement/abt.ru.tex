\gdef\thisproblemauthor{Иван Казменко}
\gdef\thisproblemdeveloper{Иван Казменко}
\begin{problem}{Садовод Абт и неправильный квадрат}
{abt.in}{abt.out}
{2 секунды}{256 мебибайт}{}

Престарелый садовод Абт живёт на окраине городка Штиль.
Накопив приличную сумму, однажды утром он отправился в мэрию
с намерением приобрести участок земли под новый сад.
Цена оказалась довольно высокой, однако ему всё же удалось
купить $n$ квадратных метров земли.
Условия сделки таковы, что теперь садоводу следует пойти домой,
выбрать прямоугольный участок нужной площади рядом со своим домом
и вызвать регистратора из мэрии.
Абт может выбрать любой прямоугольный участок размера $a \times b$ метров,
так чтобы $a$ и $b$ были целыми положительными числами и произведение
$a \cdot b$ было равно $n$.

У садовода осталось совсем мало средств на забор, поэтому он хочет,
чтобы периметр сада ($2 a + 2 b$ метров) был как можно меньше.
Он слышал, что из прямоугольников одинаковой площади наименьший периметр
имеет квадрат.
Однако Абт считает квадрат слишком правильной фигурой для своего сада.
Поэтому садовод решил, что лучше всего подойдёт <<неправильный квадрат>>:
прямоугольник, длины сторон которого отличаются ровно на один метр.

Помогите садоводу и выясните, может ли он выбрать участок в форме
<<неправильного квадрата>> площади $n$, и если может, какими будут длины
его сторон $a$ и $b$.
Помните, что длины сторон должны быть положительными целыми числами.

% new page here if \newpageafterlegend is defined, requires e-tex
\ifdefined\newpageafterlegend\newpage\fi

\InputFile

В единственной строке ввода задано целое число $n$ "--- приобретённая площадь
в квадратных метрах ($1 \le n \le 1\,000\,000\,000$).

\OutputFile

В единственной строке выведите два целых числа, разделив их пробелом.
Если Абт может выбрать участок земли в форме <<неправильного квадрата>>,
эти числа должны быть длинами сторон <<неправильного квадрата>> в метрах,
перечисленными в порядке возрастания.
В противном случае оба числа должны быть равны \texttt{-1}.

\Examples

\begin{example}
\exmp{
6
}{%
2 3
}%
\exmp{
9
}{%
-1 -1
}%
\end{example}

\Explanations

В первом примере <<неправильный квадрат>> площади $6$ квадратных метров
получится, если выбрать длины сторон равными $2$ и $3$ метрам.
Обратите внимание: длины сторон должны быть выведены в порядке
возрастания, то есть ответ <<\texttt{3 2}>> не будет считаться верным.

Во втором примере не существует <<неправильного квадрата>>
площади $9$ квадратных метров, длины сторон которого в метрах
были бы целыми числами.

\end{problem}
