\gdef\thisproblemauthor{Иван Казменко}
\gdef\thisproblemdeveloper{Иван Казменко}
\begin{problem}{Братья Дорф и письма мэра}
{dorf.in}{dorf.out}
{2 секунды}{256 мебибайт}{}

Братья Дорф живут в городке Штиль.
Когда-то они все жили в одном доме, но давно разъехались,
и каждый из них живёт в своём собственном доме.

Однажды утром каждый из братьев получил по письму.
Оказывается, новый мэр городка разослал всем жителям письма,
в которых поздравил их со своим вступлением в должность.
К сожалению, никто из братьев не смог узнать, что именно написано в письме.
Дело в том, что ни один из братьев Дорф не умеет читать.

Каждый из братьев решил переслать письмо тому из братьев,
кого он считает самым умным.
Если кто-то из них считает самым умным себя, он подумал:
<<Надо будет до завтра научиться читать "--- получу письмо завтра,
тогда и прочитаю его>>.
Другие братья подумали:
<<Ну, мой самый умный брат наверняка уже умеет читать "--- пусть он
прочитает это письмо и расскажет мне, что там написано>>.
Итак, днём все они наклеили на письма адреса своих самых умных братьев
и опустили их в почтовые ящики, а вечером эти письма опять попали на почту.

Поскольку почтальон разносит почту каждое утро, на следующий день
история повторилась: все письма опять пришли братьям по почте.
Однако, на этот раз кто-то из братьев мог получить два письма
или даже больше (если его считают самым умным двое или больше братьев),
а кто-то "--- ни одного.
Вновь все братья наклеили адреса своих самых умных братьев на все письма
и днём отнесли их в почтовые ящики.

Так продолжалось $t$ дней.
К сожалению, братья были слишком заняты другими делами, поэтому
никто из них так и не научился читать за это время.
Выясните, что происходило в $t$-й день: сколько писем каждый брат
отнёс в почтовый ящик днём через $t$ дней после того, как их прислал мэр.

% new page here if \newpageafterlegend is defined, requires e-tex
\ifdefined\newpageafterlegend\newpage\fi

\InputFile

В первой строке ввода заданы через пробел два целых числа $n$ и $t$ "---
количество братьев и количество дней
($2 \le n \le 50$, $1 \le t \le 1\,000\,000\,000$).
Во второй строке заданы через пробел $n$ целых чисел:
$v_1$, $v_2$, $\ldots$, $v_n$
($1 \le v_i \le n$).
Эти числа означают, что брат с номером $i$ считает самым умным
брата с номером $v_i$.
Для удобства братья пронумерованы целыми числами от $1$ до $n$.

\OutputFile

В единственной строке выведите $n$ чисел через пробел:
$a_1$, $a_2$, $\ldots$, $a_n$.
Каждое $a_i$ "--- это количество писем, которые брат с номером $i$
отнёс на почту днём через $t$ дней после того, как письма впервые
попали к братьям.

\Examples

\begin{example}
\exmp{
2 1
2 1
}{%
1 1
}%
\exmp{
3 2
1 2 2
}{%
1 2 0
}%
\end{example}

\Explanations

В первом примере братьев двое.
Первый считает самым умным второго, второй "--- первого.
Поскольку $t = 1$, требуется выяснить судьбу писем на следующий день.
Письмо, изначально адресованное первому брату, попадёт ко второму,
а письмо второго брата "--- к первому.
Значит, каждый брат днём вновь понесёт на почту по одному письму.

Во втором примере братьев трое.
Первый и второй братья считают самыми умными самих себя,
а третий брат "--- второго.
Значит, на следующий день после получения писем от мэра
у первого брата оказалось одно письмо "--- его собственное,
а у второго брата "--- два письма: его собственное и то,
которое ему переслал третий брат.
После этого первый и второй братья послали все письма сами себе,
а наутро получили их опять.
Значит, через два дня ($t = 2$) у первого брата оказалось одно письмо,
у второго "--- два, а у третьего "--- ни одного.

\end{problem}
