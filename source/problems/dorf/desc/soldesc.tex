\textbf{Первое решение.}
Пусть $f^k (i)$ "--- номер брата, у которого ровно через $k$ дней
окажется письмо, находящееся в данный момент у $i$-го брата.
Изначально нам известны значения $f^1 (i) = a_i$.
Сначала найдём $f^2 (i) = f^1 (f^1 (i))$ для всех $i$.
Затем получим $f^4 (i) = f^2 (f^2 (i))$.
Аналогично найдём $f^8$, $f^{16}$ и так далее.
После этого, рассмотрев двоичное представление числа $t$,
можно выяснить, как перемещаются письма ровно за $t$ дней.
Например, если $t = 13 = 1101_2 = 1 + 4 + 8$,
то $f^{13} (i) = f^{8} (f^{4} (f^{1} (i)))$.

\textbf{Второе решение.}
Можно для каждого письма вычислить предпериод и период
того пути между братьями, который оно проделает.
При этом можно избавиться от предпериодов,
сделав изначально первые $n$ действий.

\textbf{Третье решение.}
Можно представить изменения количества писем у каждого брата
за один день в виде умножения на матрицу $n \times n$.
Эту матрицу можно быстро возвести в степень $t$, как в первом решении.
