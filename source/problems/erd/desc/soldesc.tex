Прежде всего заметим, что заданные остатки от деления $d$ на
$1$, $2$, $\ldots$, $n$ однозначно определяют
$d \bmod \text{НОК} (1, 2, \ldots, n)$.
Здесь НОК "--- это наименьшее общее кратное данных чисел.
Для $n = 20$ это число равно $16 \cdot 9 \cdot 5 \cdot 7 \cdot
11 \cdot 13 \cdot 17 \cdot 19 = 232\,792\,560$.

В этой задаче есть следующая тонкость.
Пусть у нас есть два ближайших подходящих к $e$ числа "---
$d_1 \le e$ и $d_2 \ge e$ "--- и мы хотим из них выбрать ближайшее:
то, для которого модуль разности $|d_i - e|$ минимален.
Необходимо вспомнить, что ответ должен быть строго положительным,
а значит, не рассматривать вариант, меньший $e$, если $d_1 \le 0$,
даже если $|d_1 - e| \le |d_2 - e|$.

\textbf{Первое решение.}
Можно просто проверить несколько ближайших дней
на соответствие показаниям часов.
Такое решение может успевать, а может не успевать выдать ответ
за отведённое время в зависимости от того, насколько оптимально
делается проверка.
Например, если проверку на соответствие начинать с б\'{о}льших модулей,
она достаточно часто будет прерываться после первых же одного-двух модулей.

\textbf{Второе решение.}
Можно воспользоваться китайской теоремой об остатках.
Каждый следующий модуль небольшой, поэтому можно просто прибавлять
предыдущее НОК до тех пор, пока остаток не станет таким, как нужно.
Например, пусть $d \bmod 1 = 0$, $d \bmod 2 = 1$, $d \bmod 3 = 2$,
$d \bmod 4 = 1$ и $d \bmod 5 = 4$.
Начнём с $d = 0$ и будем прибавлять $1$, пока не будет выполнено
$d \bmod 2 = 1$.
Далее будем прибавлять $2$, пока не окажется, что $d \bmod 3 = 2$;
после этого $d = 5$.
Равенство $d \bmod 4 = 1$ уже верно.
Наконец, будем прибавлять $12 = \text{НОК} (1, 2, 3, 4)$
до тех пор, пока не окажется верным последнее неравенство:
$d = 5 \rightarrow 17 \rightarrow 29$.
