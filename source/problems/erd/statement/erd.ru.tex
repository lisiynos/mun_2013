\gdef\thisproblemauthor{Иван Казменко}
\gdef\thisproblemdeveloper{Иван Казменко}
\begin{problem}{Сёстры Эрд и многодневные часы}
{erd.in}{erd.out}
{2 секунды}{256 мебибайт}{}

Сёстры Эрд "--- мастерицы часовой мастерской в городке Штиль.
Они изготавливают, ремонтируют и содержат в порядке все часы городка.

В день, когда часовая мастерская сестёр открылась и начала принимать
заказы, сёстры организовали выставку.
Помимо обычных часов "--- наручных, настенных, карманных, каминных "---
каждая из сестёр выполнила особую работу: многодневные часы.
В таких часах есть дополнительная стрелка, называемая дневной,
и некоторое количество $k$ делений для неё.
Эта стрелка движется равномерно и делает полный оборот ровно за $k$ дней.
Деления пронумерованы по порядку целыми числами от $0$ до $k - 1$
включительно.
Деление $0$ совпадает с $12$-часовым делением на циферблате,
остальные деления расставлены на окружности циферблата равномерно
в порядке следования по ним дневной стрелки.
Например, если $k = 3$, деление в $1$ день совпадает с $4$-часовым,
а деление в $2$ дня "--- с $8$-часовым делением на циферблате.

Старшая сестра изготовила опытный образец таких часов, в котором $k = 1$.
Это значит, что в её часах есть ровно одно деление для дневной стрелки.
Следующая по возрасту сестра собрала часы, в которых $k = 2$,
и так далее: $i$-я по возрасту сестра сделала часы, в которых $k = i$.
Все часы начали ходить одновременно ровно в полдень того дня,
в который состоялась выставка, ознаменовавшая открытие мастерской.
В этот момент все дневные стрелки указывали на деление с номером $0$.

В мастерской стоит механический календарь, который показывает число дней,
прошедших с момента её открытия.
Однажды утром календарь сломался и стал показывать неправильное число.
К полудню сёстры починили его и теперь хотят восстановить номер дня,
который календарь должен показывать.
Они помнят, что номер приблизительно равен $e$,
а ещё "--- что точный номер строго больше нуля.

В полдень сёстры посмотрели на свои многодневные часы.
Эти часы ещё ни разу не ломались, и на каждых часах дневная стрелка
оказалась ровно на каком-то делении.
Помогите сёстрам использовать показания часов,
чтобы вычислить точный номер дня.
Если возможных номеров несколько, следует выбрать из них такой,
который бы как можно меньше отличался от $e$,
а если и таких номеров больше одного "--- минимальный из них.

% new page here if \newpageafterlegend is defined, requires e-tex
\ifdefined\newpageafterlegend\newpage\fi

\InputFile

В первой строке ввода заданы через пробел два целых числа
$n$ и $e$ "--- количество сестёр и приблизительный номер дня
($2 \le n \le 20$, $1 \le e \le 1\,000\,000\,000$).
Во второй строке заданы через пробел $n$ целых чисел:
$p_1$, $p_2$, $\ldots$, $p_n$
($0 \le p_i < i$).
Эти числа означают, что в полдень интересующего нас дня у сестры
с номером $i$ многодневная стрелка часов с $i$ делениями
указывает на деление $p_i$.
Для удобства сёстры пронумерованы целыми числами от $1$ до $n$
в порядке старшинства, начиная со старшей.
Гарантируется, что показания всех многодневных часов правдивы.

\OutputFile

В единственной строке выведите целое число $d$ "--- точный номер дня.
Это число должно соответствовать показаниям всех многодневных часов,
а также быть строго положительным.
Если возможных номеров несколько, следует выбрать из них такой,
который бы как можно меньше отличался от $e$,
а если и таких номеров больше одного "--- минимальный из них.

\Examples

\begin{example}
\exmp{
3 9
0 1 2
}{%
11
}%
\exmp{
2 11
0 0
}{%
10
}%
\end{example}

\Explanations

В первом примере сестёр трое.
Дневная стрелка первых часов указывает на единственно возможное деление $0$,
вторых "--- на деление $1$, а третьих "--- на деление $2$.
Выпишем в таблицу показания часов в первые несколько дней:

%{
%\renewcommand{\tabcolsep}{3pt}
%\begin{tabular}{|c|c|c|c|c|c|c|c|c|c|c|c|c|c|}
%\hline
%день      &  0  &  1  &  2  &  3  &  4  &  5  &
%             6  &  7  &  8  &  9  & 10  & 11  & 12  \\
%\hline
%показания & 000 & 011 & 002 & 010 & 001 & 012 &
%            000 & 011 & 002 & 010 & 001 & 012 & 000 \\
%\hline
%\end{tabular}
%}

\begin{tabular}{|c|c|c|c|c|c|c|c|c|c|c|c|c|c|}
\hline
день        &  0 &  1 &  2 &  3 &  4 &  5 &  6 &  7 &  8 &  9 & 10 & 11 & 12 \\
\hline
первые часы &  0 &  0 &  0 &  0 &  0 &  0 &  0 &  0 &  0 &  0 &  0 &  0 &  0 \\
\hline
вторые часы &  0 &  1 &  0 &  1 &  0 &  1 &  0 &  1 &  0 &  1 &  0 &  1 &  0 \\
\hline
третьи часы &  0 &  1 &  2 &  0 &  1 &  2 &  0 &  1 &  2 &  0 &  1 &  2 &  0 \\
\hline
\end{tabular}

Ближайший ко дню $e = 9$ день с нужными показаниями "--- это день $d = 11$.
Такие показания встречаются и в другие дни, например, в день $5$,
но разность $|11 - 9| = 2$ "--- минимальная из возможных.

Во втором примере сестёр двое.
Дневная стрелка каждых часов указывает на деление $0$.
Такие показания часов бывают в каждый чётный день.
Ближайших к $11$-му дню чётных дней с положительным номером "--- два:
день $10$ и день $12$.
Из них следует выбрать день с минимальным номером.

\end{problem}
